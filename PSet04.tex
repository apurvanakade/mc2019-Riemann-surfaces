\documentclass{article}

\usepackage{style/preamble}
\usepackage{style/mytikz}
\usepackage{parskip}

\newcounter{dip}

\begin{document}
  \title{Problem Set 4 - Points at $\infty$}
  \date{}
  \maketitle







\begin{mdframed}
  Corrected Theorem 4.1 from PSet 02:
  \begin{theorem}
    Let $f$ be a non-constant holomorphic function defined near $z_0 \in \bbc$.
    Suppose the Taylor series of $f$ near $z_0$ has the form
    \begin{align*}
      f(z) - f(z_0) = a_k (z - z_0)^k + a_{k+1} (z - z_0)^{k+1} + a_{k+2} (z - z_0)^{k+2} + \dots
    \end{align*}
    with $a_k \neq 0$.
    Then there exists biholomorphic functions $\psi(z)$, $\phi(z)$ (with appropriate domains) such that
    \begin{align*}
      \psi \circ f \circ \phi (z) = z^k
    \end{align*}
  \end{theorem}
\end{mdframed}



\section{Complex projective space}

Projectivization provides a technique for compactifying non-compact surfaces.
However, as we will see tomorrow, this technique sometimes creates singularities so should be used with caution.

\begin{definition}
  The \emph{complex projective space} of dimension $n$ is the set
  \begin{align*}
    \bbp^n := \set{(z_0, z_1, \dots, z_n)  \mid  z_i \in \bbc \mbox{ and not all $z_i$ are zero}}/\sim
  \end{align*}
  where the equivalence $\sim$ is defined as
  \begin{align*}
    (z_0, z_1, \dots, z_n) \sim (\lambda z_0, \lambda z_1, \dots,\lambda z_n) \mbox{ for } \lambda \neq 0 \in \bbc
  \end{align*}
  The elements of $\bbp^n$ are written as $[z_0: z_1 : \dots : z_n]$ and the $z_i$'s are called \emph{homogeneous coordinates}.
\end{definition}

\begin{qbox}
  Convince yourself that $\bbp^n$ is the space of complex lines passing through the origin in $\bbc^{n+1}$. (Such spaces are called Grassmannians.)
\end{qbox}

There is a natural embedding of $\bbc^n$ in $\bbp^{n}$ given by
\begin{align*}
  (z_0, z_1, \dots, z_{n-1}) \longmapsto [z_0 : z_1 : \dots : z_{n-1} : 1]
\end{align*}
It is easy to see that this map is injective. So we can think of $\bbc^n$ as a subset of $\bbp^n$.
Any element in $\bbp^n$ of the form $[z_0 : z_1 : \dots : z_{n-1} : z_n]$ with $z_n \neq 1$ is in the image of the above embedding, hence
\begin{align*}
  \bbp^n \setminus \bbc^n
  &= \set{ [z_0 : z_1 : \dots : z_{n-1} : 0] \mid z_i \in \bbc} \\
  &\cong \bbp^{n-1}
\end{align*}
We think of the points in $\bbp^n \setminus \bbc^n$ as the ``points at $\infty$''.
So $\bbp^n$ has ``$\bbp^{n-1}$ many'' points at $\infty$.

We are mainly interested in the space $\bbp^2$.
We will use the notation
\begin{align*}
  \bbp^2 = \set{ [z : w : t] }.
\end{align*}
The points at $\infty$ are then the points with $t = 0$ i.e. $\set{[z:w:0]}$.










\section{Homogenization}
Let $p(z,w)$ be a polynomial in two variables with complex coefficients.
And let
\begin{align*}
  S_p = \set{(z,w) \mid  p(z,w) = 0} \subseteq \bbc^2
\end{align*}
\begin{definition}
  A polynomial is \emph{homogeneous} if every monomial term in it has the same total degree.
\end{definition}
\emph{Homogenization} turns $p$ into a homogeneous polynomial $\conj{p}(z,w,t)$ in three variables, where we add powers of $t$ as needed to make each term of the same degree.
\begin{ex}
  If $p(z,w) = z^2 - w^3 - w$ then $\conj{p}(z,w,t) = z^2t - w^3 - wt^2$.
  If $p(z,w) = z^2 - w^2 - w$ then $\conj{p}(z,w,t) = z^2 - w^2 - wt$.
\end{ex}

\begin{qbox}
  Let $\conj{p}(z,w,t)$ is a homogeneous polynomial.
  Prove that $(a,b,c)$ is a root of $\conj{p}$ if and only if $(\lambda a , \lambda b, \lambda c)$ is a root of $\conj{p}$ for $\lambda \neq 0 \in \bbc$.
\end{qbox}

Hence, we can ask for solutions of the homogeneous polynomial  $p(z,w,t)$  in the projective space $\bbp^2$.

If $S_p$ is the set $\set{(z,w) \mid  p(z,w) = 0} \subseteq \bbc^2$ then denote
\begin{align*}
  \conj{S_p} := \set{[z:w:t] \mid  p(z,w,t) = 0} \subseteq \bbp^2.
\end{align*}
\begin{definition}
  $\conj{S_p}$ is called the \emph{projectivization} of $S_p$.
\end{definition}

There is a natural embedding $S_p \rightarrow \conj{S_p}$ which sends a solution $(z,w)$ of $p$ to the solution $[z:w:1]$ of $\conj{p}$.
The points in $\conj{S_p} \setminus S_p$ are called the points at $\infty$.


% \begin{qbox}
%   Describe the spaces $\bbc^1 = \set{[1:t]}$ and $\bbc^2 = \set{[1:w:t]}$ as solutions of certain polynomial equations.
%   Show that projectivizations of $\bbc^1$ and $\bbc^2$ are precisely $\bbp^1$ and $\bbp^2$ respectively.
%   \begin{align*}
%     \conj{\bbc^1} = \bbp^1 \qquad \conj{\bbc^2} = \bbp^2
%   \end{align*}
% \end{qbox}

For any homogeneous polynomial $\conj{p}$ the space $\conj{S_p}$ is compact. The proof of this is essentially the fact that closed subsets of compact sets are compact and zero sets of polynomials are closed. But because of the points at $\infty$ the argument is a bit more intricate, we won't go over the details.


\begin{qbox}
  Let $q(w)$ be a complex polynomial and let $p(z,w) = z^2 - q(w)$.
  Find the number of points at $\infty$ for $\conj{S_p}$ for the following polynomials
    \begin{enumerate}
      \item $q(w) = w + b$, where $b \in \bbc$.
      \item $q(w) = w^2 + bw + c$, where $b,c \in \bbc$.
      \item $q(w) = w^n + a_{n-1} w^{n-1}+ \dots a_1 w + a_0$, where $a_i \in \bbc$ and $n \ge 3$.
    \end{enumerate}
\end{qbox}

\begin{qbox}
  Let $p(z,w)$ be an arbitrary polynomial with homogenization $\conj{p}(z,w,t)$.
  What can you say about the number of points at $\infty$ for $\conj{S_p}$?
  Can you interpret this result in terms of limits $\lim_{z \rightarrow \infty} z/w$?
\end{qbox}

\textbf{Fact:} when $q(w)$ is a non-constant polynomial of degree $\le 3$ with distinct roots and $p(z,w) = z^2 - q(w)$ the space $\conj{S_p}$ is a Riemann surface and the projection map
\begin{align*}
    \pi: \conj{S_p} &\longrightarrow \bbp^1 \\
    [z:w:1] &\longmapsto w \\
    [z:w:0]  &\longmapsto \infty
\end{align*}
is a complex differentiable map.

\begin{qbox}
  Using the Riemann--Hurwitz formula for the projection $\pi$, find the genus of the curves $\conj{S_p}$ when $p(z,w) = z^2 - q(w)$ and $q$ is a non-constant polynomial of degree $\le 3$ with distinct roots.
\end{qbox}












\section{Fermat's conjecture for function fields}

\begin{theorem}
  The are no non-constant complex coefficient polynomial solutions to the equation
  \begin{align*}
    (x(t))^d + (y(t))^d  = (z(t))^d
  \end{align*}
  if $d > 2$, with $\mathrm{gcd}(x(t), y(t), z(t)) = 1$.
\end{theorem}

\begin{proof}
  Consider the polynomial $p(z,w) = z^d + w^d - 1$ with homogenization $p(z,w,t) = z^d + w^d - t^d$.
  $\conj{S_p}$ has exactly $d$ points at $\infty$ given by $z^d + w^d = 0$, namely
  \begin{align*}
    [\zeta_1:1:0]\:, \: [\zeta_2:1:0]\:, \dots,\: [\zeta_d:1:0] \mbox{ where } \zeta^d_i = -1
  \end{align*}
  One can show that the projection map
  \begin{align*}
      \pi: \conj{S_p} &\longrightarrow \bbp^1 \\
      [z:w:1] &\longmapsto w \\
      [z:w:0]  &\longmapsto \infty
  \end{align*}
  is a complex differentiable map, and hence a ramified covering.

  Consider a point $w \in \bbc$. The points in $\pi^{-1}(w)$ are the elements $[z:w:1]$
  such that $z^d = 1 - w^d$.
  Hence,
  \begin{enumerate}
    \item $\pi^{-1}(w)$ has size $d$ if $ 1 - w^d \neq 0$,
    \item $\pi^{-1}(w)$ has size $1$ if $ w^d = 0$,
    \item $\pi^{-1}(\infty)$ has size $d$.
  \end{enumerate}

  This tells us that there are $d$ branch points given by the $d^{th}$ roots of unity, call them $\tau_1, \dots, \tau_d$, and the fiber over each branch point is single element $[0:\tau_i:1]$.

  Plugging this in the Riemann--Hurwitz formula we get
  \begin{align*}
    \chi(\conj{S_p})
    &= d \cdot \chi(\bbp^1) - \sum_{d} (d - 1) \\
    &= 3d - d^2
  \end{align*}

  Hence if $d > 2$, $\chi(\conj{S_p}) < 2$ and hence genus of $\conj{S_p} > 0$.
  Hence if $d > 2$, there are no non-constant complex differentiable maps
  \begin{align*}
    \bbp^1 \longrightarrow \conj{S_p}
  \end{align*}

  But a solution $(x(s),y(s),z(s))$ to the equation $x^d + y^d = z^d$ defines a complex differentiable map
  \begin{align*}
    \bbp^1 &\longrightarrow \conj{S_p} \\
    s &\longmapsto [x(s) : y(s) : z(s)]
  \end{align*}
  extended to $\infty$ by taking the limit. This map would be non-constant if $\mathrm{gcd}(x,y,z) = 1$. But no such map exists.
\end{proof}

\end{document}
