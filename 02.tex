\section{Compactification}
\begin{figure}
  \centering
  \begin{tikzpicture}
  [dip/.style={/utils/exec=\stepcounter{dip},
    insert path={%
      to[out=0,in=180]
      ++(0.25,#1) node[bullet](dip-\the\value{dip}){}
      to[out=0,in=180] ++(0.25,-1*#1)
      }}, bullet/.style={circle,fill,inner sep=1.5pt}]
  \begin{scope}[thick,local bounding box=dips]
   \draw (1,2.5) -- (5.25,2.5) [dip=-2.5mm]-- (7,2.5);
      \node[right=3pt of dip-1]{$a$};
   \draw (1,2)  -- (2.25,2) [dip=-2.5mm] -- (5.25,2) [dip=2.5mm] --(7,2);
      \node[right=3pt of dip-2]{$b$};
   \draw (1,1.5) -- (2.25,1.5) [dip=2.5mm] --(5.25,1.5) [dip=-5mm]  -- (7,1.5);
      \node[above right=-1.5pt and 5pt of dip-5]{$c$};
   \draw (1,1) -- (7,1);
   \draw (1,0.5) --(5.25,0.5) [dip=5mm] -- (7,0.5);
  \end{scope}
    \node[left=2pt of dips.west] (X) {$X$};
  \draw (7,-0.5) -- (1,-0.5)  node[left=2pt] (Y) {$Y$};
  \draw[<-] (Y) -- (X) node[left, midway] {$f$};
    \foreach \X in {1,2}
     {
      \draw[dashed] (dip-\X|-Y) node[bullet]{} -- (dip-\X|-0,2.75);
     }
\end{tikzpicture}

  \caption{Not sure how to adjust width in this case.}
\end{figure}
We saw that the Riemann surface $\Gamma_{p(z)}$ is homeomorphic to $\bbc$.

We can generalize the previous constructions to studying graphs of
\begin{align*}
  p(z) &= w^p \\
  p(z,w) &= 0
\end{align*}
But before we do this, we need to make one more leap which is again inspired from topology.

The surface $\Gamma_{p(z)}$ costructed above is not compact.
Non-compact objects are harder\footnote{In most case, but of course not always.} to study than compact ones because the behaviour of limits is harder to control for non-compact surfaces and functions on non-compact .
\begin{qbox}(Optional exercise)
  Let $X$ be a subset of $\bbc$.
  \begin{enumerate}
    \item Show that if $X$ is compact then every function $f:X \rightarrow \bbr$ is bounded.
    \item If $X$ is open, find an unbounded continuous function $f:X \rightarrow \bbc$.
  \end{enumerate}
\end{qbox}
Furthermore, non-compact subsets come in all shapes and sizes (for example, every open subset of $\bbc$ is non-compact) but as we'll see in the next section there are various classification theorems for compact ones.











\subsection{Behaviour at $\infty$}
