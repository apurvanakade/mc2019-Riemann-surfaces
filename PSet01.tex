\documentclass{article}

\usepackage{style/preamble}
\usepackage{style/mytikz}
\usepackage{parskip}

\begin{document}
  \title{Problem Set 1 - The Riemann Sphere}
  \date{}
  \maketitle

    \begin{mdframed}
      Corrected open mapping theorem statement:
      \begin{theorem}[Open mapping theorem]
        \label{th:OpenMappingTheorem}
        If $f:U \rightarrow \bbc$ is a \emph{non-constant} complex differentiable then for any open set $V \subseteq U$, the set $f(V)$ is an open subset of $\bbc$.
      \end{theorem}
    \end{mdframed}

    \section*{Holomorphic functions on $\bbp^1$}
    A complex differentiable function $f: X \rightarrow \bbc$ is called a \emph{holomorphic} function on $X$.

    The Riemann sphere $\bbp^1$ is the set $\bbc \cup \set{\infty}$.
    We write $\bbp^1$ as the union of two sets
    \begin{align*}
      U_1 = \bbc && U_2 = \bbc \setminus \set{0} \cup \set{\infty}
    \end{align*}
    The set $U_1$ is the standard complex plane, but the set $U_2$ is not.
    We can turn $U_2$ into the complex plane by using the following function
    \begin{align*}
      \varphi_2 : U_2 &\longrightarrow \bbc \\
      z &\mapsto z^{-1} \mbox{ if } z \neq \infty \\
      \infty &\mapsto 0
    \end{align*}
    Thus we can think of $\bbp^1$ as two copies of the complex plane ($U_1$ and $\varphi_2(U_2)$) glued together.

    A function $f: \bbp^1 \rightarrow \bbc$ is defined as a pair of functions
    \begin{align*}
      f_1 : U_1 \rightarrow \bbc && f_2 : U_2 \rightarrow \bbc
    \end{align*}
    such that $f_1$ and $f_2$ agree on $U_1 \cap U_2$.

    We can only make sense of the complex differentiable functions when both the source and target are open subsets of $\bbc$. For this reason, we define holomorphic functions on $\bbp^1$ as follows.

    A function $f: \bbp^1 \rightarrow \bbc$ is holomorphic if
    \begin{enumerate}
      \item $f|_{U_1}$ is holomorphic,
      \item $f \circ \varphi_2^{-1}|_{\varphi(U_2)}$ is holomorphic.
    \end{enumerate}
    \begin{equation*}
      \begin{tikzcd}
        U_1 \ar[r,"f|_{U_1}"] &  \bbc\\
        U_2 \ar[d, "\varphi_2"] \ar[r, "f"] & \bbc \\
        \bbc \ar[u, "\varphi_2^{-1}", bend left] \ar[ur, "f \circ \varphi_2^{-1}", swap, dashed]
      \end{tikzcd}
    \end{equation*}

    \begin{qbox}
      Check that defining a function $f: \bbp^1 \rightarrow \bbc$ is equivalent to defining a pair of functions
      \begin{align*}
        f_1: U_1 &\longrightarrow \bbc \\
        f_2 \circ \varphi_2^{-1}: \varphi(U_2) &\longrightarrow \bbc \\
      \end{align*}
      such that
      \begin{align*}
        f_1(z) = f_2 \circ \varphi_2^{-1}(z^{-1}) \mbox{ whenever } z \neq 0.
      \end{align*}
      Note that the source of both $f_1$ and $f_2 \circ \varphi_2^{-1}$ is $\bbc$.
    \end{qbox}

    It is kinda hard to come up with examples because of the following theorem.

    \begin{theorem}
      The only holomorphic functions on $\bbp^1$ are the constant functions.
    \end{theorem}
    The proof is in the following exercises.

    A subset $V$ of a topological space $X$ is compact if it has the following property.
    \begin{quote}
      Every infinite sequence has a convergent subsequence i.e. for every infinite sequence of points $a_1, a_2, \dots $ in $V$ there exists a subsequence $a_{n_1}, a_{n_2}, a_{n_3}, \dots$ which converges to a point in $V$.
    \end{quote}
    % For subsets of $\bbr^n$ or $\bbc^n$ this is equivalent to saying that $X$ is closed and bounded.

    \begin{qbox}
      Prove that compact subsets $V$ of a (nice) topological space $X$ are closed i.e. if a sequence of points $a_1, a_2, \dots $ in $V$ converges to point $a \in X$ then $a$ is in $V$.
    \end{qbox}

    \begin{qbox}
      Prove that every compact subset $V$ of $\bbc$ is bounded i.e. there exists a real number $M$ such that $z < M$ for all $z \in V$.
    \end{qbox}

    \begin{qbox}
      \label{q:imageOfCompact}
      Use the fact that every infinite sequence has a convergent subsequence to argue that for any continuous function $g: X \rightarrow Y $ the image of a compact set is compact.
    \end{qbox}

    Assume the following fact:
    \begin{quote}
      The Riemann sphere is a compact topological space.
    \end{quote}
    One way to see this is that the Riemann sphere is topologically isomorphic to the sphere $S^2$ in $\bbr^3$ (hence the name) which is a closed and bounded subset of $\bbr^3$. It is not hard to show that such subsets are compact.

    \begin{qbox}
      Argue that the image of any continuous function $f: \bbp^1 \rightarrow \bbc $ is bounded.
    \end{qbox}

    \begin{qbox}
      Using Liouville's theorem, argue that if $f$ is a holomorphic function on $\bbp^1$ then $f|_{U_1}$ is a constant function.
    \end{qbox}
    \begin{qbox}
      Using continuity, argue that if $f$ is a holomorphic function on $\bbp^1$ then $f$ is a constant function.
    \end{qbox}









    \section*{Meromorphic functions on $\bbc$}
    A complex differentiable function $f: X \rightarrow \bbp^1$ is called a \emph{meromorphic} function on $X$.

    We can only make sense of the complex differentiable functions when both the source and target are open subsets of $\bbc$. For this reason, we define meromorphic functions on $X$ as follows.

    A function $f: X \rightarrow \bbp^1$ is meromorphic if
    \begin{enumerate}
      \item $f$ is holomorphic when restricted to $f^{-1}(U_1)$,
      \item $\varphi_2 \circ f$ is holomorphic when restricted to ${f^{-1}(U_2)}$.
    \end{enumerate}
    \begin{equation*}
      \begin{tikzcd}
        f^{-1}(U_1) \ar[rr,"f"] & &  U_1 = \bbc\\
        f^{-1}(U_2) \ar[rr, "f"] \ar[rrrr, "\varphi_2 \circ f", swap, bend right] & & U_2 \ar[rr, "\varphi_2"] & & \bbc
      \end{tikzcd}
    \end{equation*}

    It gets tedious to keep track of all the inverses and the sources and targets.
    We use the following shorthand notation to simplify the clutter.
    Let $\varphi_1: U_1 \rightarrow \bbc$ be the identity function, $\varphi_1(z) = z$.
    Then a function $f: X \rightarrow \bbp^1$ is meromorphic if the two functions
    \begin{enumerate}
      \item $\varphi_1 \circ f$,
      \item $\varphi_2 \circ f$,
    \end{enumerate}
    are holomorphic wherever they make sense.
    \begin{equation*}
      \begin{tikzcd}
        f^{-1}(U_1) \ar[rrrr, "\varphi_1 \circ f", swap, bend right] \ar[rr,"f|_{f^{-1}(U_1)}"] &  &  U_1 \ar[rr, "\varphi_1"] & & \bbc\\\\
        f^{-1}(U_2) \ar[rr, "f|_{f^{-1}(U_2)}"] \ar[rrrr, "\varphi_2 \circ f", swap, bend right] &  & U_2 \ar[rr, "\varphi_2"] & & \bbc
      \end{tikzcd}
    \end{equation*}

    \begin{qbox}
      Which of the following functions are meromorphic functions on $\bbc$?
      \begin{enumerate}
        \item $f(z) = z$
        \item \begin{align*}
            f(z) = \begin{cases}
                    z^{-1} & \mbox{ if } z \neq 0\\
                    \infty & \mbox{ if } z = 0
                  \end{cases}
          \end{align*}
        \item \begin{align*}
            f(z) = \begin{cases}
                    e^{1/z} & \mbox{ if } z \neq 0\\
                    \infty & \mbox{ if } z = 0
                  \end{cases}
          \end{align*}
      \end{enumerate}
    \end{qbox}










    \section*{Meromorphic functions on $\bbp^1$}
    \begin{qbox}
      Show that a function $f: \bbp^1 \rightarrow \bbp^1$ is meromorphic if the four functions
      \begin{enumerate}
        \item $\varphi_1 \circ f \circ \varphi_1^{-1}$,
        \item $\varphi_1 \circ f \circ \varphi_2^{-1}$,
        \item $\varphi_2 \circ f \circ \varphi_1^{-1}$,
        \item $\varphi_2 \circ f \circ \varphi_2^{-1}$,
      \end{enumerate}
      are holomorphic wherever they make sense.
    \end{qbox}

    \begin{qbox}
      Let $p(z)$ and $q(z)$ be polynomials with no common roots. Assume that $q(z)$ is not the 0 polynomial.

      Show that the following function is a meromorphic function on $\bbp^1$.
      \begin{align*}
        f(z) = \begin{cases}
          \dfrac{p(z)}{q(z)} & \mbox{ if } z \neq \infty, q(z) \neq 0, \\
          \infty & \mbox{ if } z \neq \infty, q(z) = 0, \\
          \lim \limits_{z \rightarrow \infty} \dfrac{p(z)}{q(z)} & \mbox{ if } z = \infty.
      \end{cases}
      \end{align*}
      Such a function is called a \emph{rational function}. It is common to simply write $f(z) = \dfrac{p(z)}{q(z)}$.
    \end{qbox}

    Turns out these are all the meromorphic functions on $\bbp^1$.
    \begin{theorem}
      Every meromorphic function on $\bbp^1$ is a rational function.
    \end{theorem}
    The following exercises provide the proof of this theorem.

    Let $f: \bbp^1 \rightarrow \bbp^1$ be a meromorphic function.
    Let $\calz = f^{-1}(0) \cap \bbc$ and $\calp = f^{-1}(\infty) \cap \bbc$. $\calz$ is called the set of zeroes and $\calp$ is called the set of poles.
    \begin{qbox}
      Using the isolated zeroes property of complex differentiable functions argue that both the sets $\calz$ and $\calp$ are isolated i.e. for every point $x \in \calz$ there exists a neighborhood  $U$ of $x$ such that $U \cap \calz = \set{x}$. Similarly, for $\calp$.
    \end{qbox}
    \begin{qbox}
      Using the fact that every infinite sequence in a compact set has a convergent subsequence, and that $\bbp^1$ is compact, argue that both $\calz$ and $\calp$ are finite sets.
    \end{qbox}
    Let $\calz = \set{z_1, \dots, z_m}$ and $\calp = \set{p_1, \dots, p_n}$.
    Assume the following fact for now. We'll prove it in class tomorrow.
    \begin{quote}
      The function
      \begin{align*}
        g(z) = f(z) \cdot \dfrac{(z-p_1)^{k_1} \dots (z-p_n)^{k_n}}{(z-z_1)^{\ell_1} \dots (z-z_m)^{\ell_m}}
      \end{align*}
      is meromorphic and has no zeroes or poles, for some positive integers $k_1, \dots, k_n$ and $\ell_1, \dots, \ell_m$.
    \end{quote}
    \begin{qbox}
      Check that the open mapping theorem \ref{th:OpenMappingTheorem} extends verbatim to meromorphic functions on $\bbp^1$.
    \end{qbox}
    \begin{qbox}
      Using the open mapping theorem and Q.\ref{q:imageOfCompact} argue that $g$ is either a constant function or the image of $g$ is all of $\bbp^1$.\hint{You will need to use the fact that the only open and closed subsets of $\bbp^1$ are the empty set and $\bbp^1$ itself.}
    \end{qbox}
    Because the only zero or pole of $g$ is at $\infty$ (which can be one or the other) the image of $g$ cannot be all of $\bbp^1$. Hence, $g$ is a constant function i.e. $g(z) = c$ for some $c \in \bbc$. Hence,
    \begin{align*}
      &&
      f(z) \cdot \dfrac{(z-p_1)^{k_1} \dots (z-p_n)^{k_n}}{(z-z_1)^{\ell_1} \dots (z-z_m)^{\ell_m}}
      = c \\
      \implies
      &&
      f(z) = c\dfrac{(z-z_1)^{\ell_1} \dots (z-z_m)^{\ell_m}}{(z-p_1)^{k_1} \dots (z-p_n)^{k_n}}.
    \end{align*}
\end{document}
