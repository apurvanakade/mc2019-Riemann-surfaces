\section{Introduction}
We know from real analysis that is difficult to define inverses of functions, mainly because many functions of interest are not bijective.
But inverses \emph{are} very important.
\begin{center}
  \begin{tabular}{l | l}
    Function & Inverse \\\hline
    $x^2$ & $\pm \sqrt{x}$ \\
    $x^n$ & $x^{1/n}$ \\
    $e^x$ & $\ln x$
  \end{tabular}
\end{center}
We use various tricks to define the inverse. For example, for definiting the inverse of $f(x) = x^2$ we can
\begin{enumerate}
  \item Restrict the domain of the inverse to non-negative integers and \emph{choose} $\sqrt{x}$ to always be positive.
  \item Study the graph $y = x^2$ instead, thereby bypassing the need to define an inverse explicitly.
\end{enumerate}
Both of these methods generalize to (nice) complex functions, with the first giving rise to the notion of branch cuts and branch points and the second giving us Riemann surfaces.
